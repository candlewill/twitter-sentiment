%\documentclass[12pt]{article}
\documentclass[final,3p,12pt]{elsarticle}
\usepackage{enumerate}
\usepackage{courier}
\usepackage{graphicx}
\usepackage{hyperref}
\usepackage{float}
\usepackage{setspace}
\usepackage{fancyvrb}
\usepackage[numbers]{natbib}

\onehalfspacing
%\doublespacing
\restylefloat{table}

\begin{document}

\thispagestyle{empty}
\begin{center}

\huge

Utilizing Maximum Entropy Classification for Sentiment Analysis in Twitter Messages

\large
\vspace{2.0cm}
By

\vspace{1.0cm}
Douglas Anderson

0703557

dander01@uoguelph.ca

\vspace{1.5cm}
CIS*4900 - Undergraduate Research Project

August 4th, 2014 %TODO fix this

\vspace{1.5cm}

Supervisor:

Professor Fei Song

\vspace{1.5cm}

Second Reader:

Professor Xining Li

\vspace{1.5cm}

Draft 3 %TODO ditch this

\end{center}

\newpage
\setcounter{page}{1}

\section{Introduction}
\label{section:intro}

Social media allows people to discuss topics and disseminate their opinions
with a near negligible cost. The low barrier of entry to social media allows
everyone to share their opinions as much as they please. The scale of textual
data from social media is enormous. In order to gain insight into the `overall'
opinion on particular topics, the process of determining sentiment must be
automated.

Performing sentiment analysis is becoming a very important field of study for
many businesses. If a company can determine what consumers think of their
products they can take actions such as improve their products or appease an
unsatisfied customer.

Sentiment analysis is simply a classification task that attempts to label
fragments of text that appropriately summarizes the attitude of the text.
Natural Language Processing (NLP) provides a number of method for preforming
sentiment analysis. Many of these methods, such as Naive Bayes and Maximum
Entropy, take a probabilistic approach to performing the task. Another commonly
used method of performing sentiment analysis is the uses of a support vector
machine (SVM).

\subsection{Specific Problem}
\label{ssection:specificproblem}

The focus of this undergraduate research project has been to preform sentiment
analysis in data from the microbloging website Twitter. All of the messages,
known as Tweets, are constrained in length to 140 characters. This limit forces
many users of Twitter to use abbreviations and contractions. This data presents
some interesting problems to performing good sentiment analysis due to it's
short and informal nature. Twitter makes it extremely easy to publish a Tweet
and has no way to edit a Tweet once published. This results in more spelling
mistakes and inconsistent grammar then other sources of text. Tweets often also
contain URLs to other websites. Tweets also contain a certain amount of OOV
(Out-Of-Vocabulary) words, such as Hashtags, a tagging system for topics
allowing Tweets in a similar vein of conversation to be found. Other OOV words
include mentions (e.g. `@BarackObama') which are a method to direct a Tweet to
one or more users, as well as `RT' which is an acronym for `retweet'
communicating that this particular Tweet is a reissuing of another users Tweet.
Another issue that is out of the scope of this project is that Twitter
currently supports 35+ languages, meaning that in-order to determine the
sentiment most systems must first determine which language the Tweet is in.

\subsection{Task Description}
\label{ssection:taskdescription}

Organizers of the SemEval-2013 workshop created various sentiment analysis
tasks to be preformed. The task focused on for this project was task-2b:
Message polarity classification. The organizers used the Amazon service
``Mechanical Turk" to get many humans to annotate English language Tweets as
`positive', `negative' or `neutral'. In Tweets with mixed sentiments, the
prevailing sentiment was chosen. Participants were tasked with creating a
classifier to establish labels for Tweets not in the training dataset. The
various submissions of the participants were then evaluated by the average of
the F-positive and the F-negative. The submissions were allowed categorized as
`unconstrained' or `constrained' depending on if they did or did not use
supplemental data, respectively.

\section{Background}
\label{section:background}

According to `Opinion Mining and Sentiment Analysis' \cite{Pang2008}: ``The
year 2001 or so seems to mark the beginning of widespread awareness of the
research problems that sentiment analysis and opinion mining raise''. This
explosion in research has happened very recently, and as such the field is
moving very quickly. Much of the earlier work in the field involved data from
other contexts such as Movie reviews \cite{Pang2002}, and blog posts
\cite{Melville2009}. The challenges associated with classifying Twitter data
has only attracted researchers in the past few years \cite{Jianfeng2013}
\cite{Barbosa2010} \cite{Gokulakrishnan2012}.

Sentiment analysis is a very specific type of \textbf{text classification}, a
problem that is well summarized in chapter 6 of \textit{Mining Text Data} by
Aggarwal and Zhai \cite{Aggarwal2012} as:

\begin{quote}
    We have a set of training records $\mathcal{D} = \{X_{1} , . . .\ , X_{N}
    \}$, such that each record is labeled with a class value drawn from a set
    of k different discrete values indexed by $\{1 . . . k\}$. The training
    data is used in order to construct a classification model, which relates
    the features in the underlying record to one of the class labels.
\end{quote}

In the case of sentiment analysis the class labels being applied to the records
denote the attitude or opinion in the text. The focus of this project is
determining sentiment at the document level which assumes that the document
expresses a predominate sentiment. This is a somewhat reasonable assumption
given the fact that Tweets are limited in size, making it difficult to express
mixed sentiment. Sentiment analysis is very closely linked with opinion mining,
which is the practice of extracting sentiment on a particular topic in order to
gain insights into public opinion on the topic.

Much of the previous research make use of Naive Bayes, Maximum Entropy, and
SVM classifiers. The Naive Bayes classifier uses the occurrences of features in
the \textit{training set} to calculate the probability that an example in the
\textit{test set} belongs in a given class. This method has the benefit of
linear time training since there is no iterative process, but I makes the
assumption that all features are independent, which is not the case in natural
language. The Maximum Entropy approach is explained in detail in section
\ref{ssection:summaryofmaxent}. While Maximum Entropy and Naive Bayes are both
probabilistic approaches the to problem, Support Vector Machine (SVM)
classifiers take a different approach to the problem. ``SVM Classifiers attempt
partition the data space with the use of linear or non-linear delineations
between the different classifiers.''\cite{Aggarwal2012}.

In order to perform text classification of any kind one of the most important
aspects is deciding how to represent a document and how to select
representative features. A document is often represented as either a
\textit{bag-of-words} or \textit{string}. For the limited time frame of this
project the \textit{bag-of-words} method was used.

There are many established methods of selecting features such as document
frequency, which is simply the number of documents that a word occurs at least
once in. Document frequency has the added benefit of not requiring a class
labels to select features. While document frequency is the only method employed
for this project many other method exist, such as Gini Index, Information Gain,
and $\chi^{2}$-statistic \cite{Aggarwal2012}. Each of these more complex
methods rely on the class label in the training set in order to determine how
well the feature discriminates between classes.

\section{Our Implementation}
\label{section:implementation}

\subsection{Summary of Maximum Entropy}
\label{ssection:summaryofmaxent}

The Maximum Entropy classification method was focused on due to the limited
duration of the project, however a Naive Bayes classifier was also implemented
to provide a point of comparison. This supervised machine learning method
calculates the probability that a given document $d$ belongs to the class $c$.
The Maximum Entropy approach is summarized below. For a more detailed
description, see: Nigam et al. \cite{Nigam1999}.

The training data is used to establish constraints in the following fashion:

\begin{equation}
    P(c | d) = \frac{1}{Z(d)} \mathrm{exp}(\sum\limits_{i} \lambda _{i,c} F_{i,c}(d,c) )
\end{equation}

where $Z(d)$ is a normalization function, $F_{i,c}$ is a feature function for
feature $f_{i}$ and class $c$, and $\lambda_{i}$ is a feature parameter.

In order to establish reasonable $\lambda_{i}$ for each feature the Improved
Iterative Scaling (IIS) is employed. Given a set of training data
$\mathcal{D}$, IIS trains a model $\Lambda$ by performing a hillclimbing
procedure with the function $l(\Lambda|\mathcal{D})$, where:

\begin{equation}
    l(\Lambda | \mathcal{D}) = \sum\limits_{d\in\mathcal{D}} \sum\limits_{i}
    \lambda_{i} f_{i}(d,c(d)) - \sum\limits_{d\in\mathcal{D}} \mathrm{log}
    \sum\limits_{c} \mathrm{exp} \sum\limits_{i} \lambda_{i} f_{i}(d,c)
\end{equation}

Starting with any initial vector of parameters $\Lambda$, at each iteration
we improve the parameters by setting $\Lambda = \Lambda + \Delta$.

% TODO expand on IIS

\subsection{Preprocessing and Feature Selection}
\label{ssection:featureselection}

The classifier produced for this project is implemented in \textit{python} and
make extensive use of the \textit{nltk} (Natural Language Toolkit) library. As
well the implementation makes use of the \textit{flex} tool to create a lexer
for tokenization. The classifier does not make uses of any supplemental
datasets and therefore should be considered constrained within the context of
the task. The classifier adheres to the following procedure:

\begin{enumerate}

    \item \textbf{Tokenization} - The text is read into memory and separated into
        tokens. Each token represents an atom of text. Token types include
        word, number, user, hashtag, url, emoticon and punctuation.

    \item \textbf{Normalizing} - Each token is simplified to assist in
        classification by reducing the number of tokens.

        \begin{itemize}

            \item \textbf{Words} are coerced into lowercase

            \item \textbf{Numbers} are replaced with "\#NUM"

            \item \textbf{User} are replaced with "@USER"

            \item \textbf{Hashtags} are replaced with "\#OCTOTHORPE"

            \item \textbf{URLs} are replaced with "@URL"

            \item \textbf{Emoticon} are grouped into four different groups
                based on their intended meaning. Each group is then replaced
                with a representation. (e.g. `:)' $\rightarrow$ `EM\_HAPPY' and
                `;-p' $\rightarrow$ `EM\_WINK')

            \item most \textbf{Punctuation} is replaced with "PUNCT" however
                question marks, punctuation marks, and ellipsis are each
                replaced with their own representation.

        \end{itemize}

    \item \textbf{Feature Selection} - The number of tokens retained as
        features is further reduced.
        \begin{itemize}

            \item By \textbf{Removing Stopwords}, words that do not
                discriminate between classes are eliminated. Because of the
                limited size of the dataset only 28 stopwords are used. For a
                full list of stopwords see section \ref{ssection:stopwords}

            \item Then \textbf{Uncommon words} are removed. Words that do not
                meet a certain document frequency threshold are eliminated.
                Many of these words are proper nouns that, if retained are
                likely to cause the classifier to overfit to the training set.
                A threshold of one was used throughout the testing process.

        \end{itemize}

    \item \textbf{Splitting} - The corpus of documents are random separated
        into three sets: the \textit{training set}, the \textit{validation set},
        and the \textit{testing set}. Respectively, each dataset is 60\%, 20\%,
        and 20\% of the corpus.

    \item \textbf{Training} - The classifier is trained using the
        \textit{training set}.  The classifier uses the \textit{validation set}
        to evaluate the training process and attempt to curtail overfitting.

    \item \textbf{Testing} - The performance of the dataset is evaluated
        against the \textit{testing set}.

\end{enumerate}


\subsection{Training Optimization}
\label{ssection:optimization}

The implementation created for this project used a \textit{validation set} during
training as the primary mechanism to avoid overfitting. While training the
effectiveness of the classifier was evaluated after every iteration, using the
accuracy or F-positive measure. During the training process the following
procedure was followed:

\begin{enumerate}

    \item The classifier was trained for 5 iterations to establish reasonable
        initial weights.

    \item Evaluate the classifier using the validation set. Like the
        \textit{testing set} that will evaluate the classifiers final
        performance, the \textit{validation set} shares no common examples with
        the \textit{training set} and will therefore give a reasonable
        indication of how the classifier will perform with unseen examples.

    \item Repeat the following until performance degrades for a specified
        number of iterations

    \begin{enumerate}

        \item Train the classifier for another iteration.

        \item Evaluate performance with the \textit{validation set}.

    \end{enumerate}

    \item Restore the $\Lambda$ value of the best performing iteration.

\end{enumerate}

\section{Experiments and Discussion}
\label{section:experiments}

\subsection{Dataset}
\label{ssection:dataset}

The dataset for the task consists of 6225 labelled Tweets. 36.66\% of the
corpus has `positive' label, 14.04\% has a `negative' label, 49.30\% has a
label that conveys neutral sentiment. This neutral label is one of `neutral',
`objective', or `objective-OR-neutral'. The following is an example of a
positive Tweet:

\begin{Verbatim}[fontsize=\footnotesize]
"positive", "Gas by my house hit $3.39!!!! I'm going to Chapel Hill on Sat. :)"
\end{Verbatim}

\subsection{Evaluation Measures}
\label{ssection:evaluation}

After the SemEval-2013 workshop the organizers released a paper discussing the
proceedings and the results of the task \cite{Nakov2013}. The paper establishes
baseline in terms of $F$. $F$ can be calculated as:

\begin{equation}
    F = \frac{2 \frac{P_{pos}R_{pos}}{P_{pos} + R_{pos}} +
    2 \frac{P_{neg}R_{neg}}{P_{neg} + R_{neg}}}{2}
\end{equation}

Which can be simplified to:

\begin{equation}
    F = \frac{P_{pos}R_{pos}}{P_{pos} + R_{pos}} + \frac{P_{neg}R_{neg}}{P_{neg} + R_{neg}}
\end{equation}

The \textit{Precision} measure for the positive label is $P_{pos} =
\frac{\mathrm{\# of\ true\ postive}}{\mathrm{\# of\ true\ postive} + \mathrm{\#
of\ false\ postive} }$. The \textit{Recall} measure for the positive label is
defined as $R_{pos} = \frac{\mathrm{\# of\ true\ postive}}{\mathrm{\# of\ true\
postive} + \mathrm{\# of\ false\ negative} }$. $P_{neg}$ and $R_{neg}$ can be
calculated in a similar fashion.

\subsection{Improving Performance Measurement}
\label{ssection:performancemeasurement}

    Testing the performance of the system with a single run can produce skewed
    results since whether `easy' Tweets end up in \textit{training set} or the
    \textit{test set} will affect performance. To combat this the following
    \textbf{exhaustive cross-validation} method is used:

        \begin{enumerate}

        \item \textbf{Partitioning} - The corpus of documents are shuffled and
            separated into 5 or 10 sections.

        \item \textbf{Folding} - The training and testing process are run
            multiple times with sections used for various roles. The roles are
            permuted so that the classifier is trained and tested with the
            sections in all possible configurations. In each fold 60\% of the
            partitions are designated as part of the \textit{training set},
            20\% as the \textit{validation set}, and 20\% as the \textit{test set}.
            Since the Naive Bayes classifier does not use a validation set
            40\% of the partitions are designated as part of the \textit{test set}

            %TODO is this clear?

        \begin{enumerate}

            \item \textbf{Training} - The classifier is trained with the
                partitions designated as part of the \textit{training set} for
                the fold. The Maximum Entropy classifier also uses the
                \textit{validation set} to determine when to stop the training
                process.

            \item \textbf{Testing} - The performance of the classifier is
                evaluated using the \textit{test set} and the results are recorded.

        \end{enumerate}

        \item \textbf{Aggregating} - The results from all folds are averaged.

        \end{enumerate}


\subsection{Baseline}
\label{ssection:baseline}

The baseline for this project is established by running the Maximum Entropy
classifier without any stopword removal, uncommon word removal (words with a
document frequancy > 1), Or feature normalization. The F-score that is achieved
is \textbf{50.72}.

\subsection{Experimental Results}
\label{ssection:results}

The following experiments examine the results of the Maximum Entropy classifier
and the Naive Bayes classifier with various feature selection methods turned on
or off. The `Baseline' experiment refers to no feature selection. The `Just
Stopwords' experiment refers the only removing stopwords. The `Just Uncommon'
experiment refers to only removing words with a document frequency of one. The
`Just Normalization' experiment refers to only applying feature normalization
mechanisms. The `All' experiment refers to testing the classifier with all of
the previous feature selection methods turned on.

\subsubsection{Number of features}
\label{ssection:numfeatures}

Number of features retained after feature selection varies from Tweet to Tweet.
To summarize the number of features retained after different process Table
\ref{table:numfeatures} presents the mean number of features per Tweet for
positive, neutral, negative examples as well as an overall average.

\begin{table}[H]
    \begin{center}
    \begin{tabular}{|c||r|r|r|r|}
        \hline
        Description        & Positive & Neutral & Negative & Overall \\
        \hline
        All                & 14.85    & 14.06   & 15.10    & 14.49 \\
        Just Stopwords     & 17.13    & 16.86   & 17.41    & 17.04 \\
        Just Uncommon      & 18.43    & 17.87   & 18.85    & 18.21 \\
        Just Normalization & 19.65    & 19.16   & 20.06    & 19.46 \\
        Baseline           & 20.65    & 20.45   & 20.96    & 20.58 \\
        \hline
    \end{tabular}
    \end{center}
    \caption{`Positive', `Neutral', `Negative', and `Overall' refer to the
    average number of features per Tweet in each experiment.}
    \label{table:numfeatures}
\end{table}

%TODO check these results

As presented in Table \ref{table:numfeatures}, the number of features decreases
each time a feature set simplification method is turned on. Normalization does
not appear to significantly reduce the number of features per Tweet. This make
sense because the number of features in would only be reduced if it contained
tokens that were different before simplification and were the same after (e.g.
A Tweet that contained `2' and `3' before normalization would only contain the
feature `\#NUM' after normalization).

%TODO clarify

\subsubsection{Maximum Entropy Classifier}

\begin{table}[H]
    \begin{center}
    \begin{tabular}{|r||c|c|c|}
        \hline
        Description    & $\mu F$ & $\sigma F$ & $\delta\ \mu F$ \\
        \hline
        All            & \textbf{50.96} & 1.93 &  2.59 \\
        Just Stopwords & 48.27   & 1.83 & -0.10 \\
        Just Uncommon  & 49.02   & 1.96 &  0.65 \\
        Just Normalize & 49.55   & 2.10 &  1.18 \\
        Baseline       & 48.37   & 2.15 & - \\
        \hline
    \end{tabular}
    \end{center}
    \caption{Performance of the Maximum Entropy classifier using exhaustive
    cross-validation with 5 partitions}
    \label{table:maxent-results}
\end{table}

As Table \ref{table:maxent-results} shows `All' experiment provides the best
results over the baseline. The `Just Stopwords' experiment actually provides
slightly worse results then the baseline indicating that in context of Tweets
with so few features in each example every feature counts. The `Just Uncommon'
experiment does not provide very significant gains when used alone. This is
potentially due to the fact that with out normalization turned, features such
as `just' are `Just' are determined to be different and may be removed if each
only occur once.

\begin{figure}[H]
    \begin{center}
    \includegraphics[scale=0.75]{figures/maxent_boxplot.png}
    \caption{This boxplot shows the F-score of each of twenty folds in all five
    Maximum Entropy experiments}
    \label{figure:maxent}
    \end{center}
\end{figure}

\subsubsection{Naive Bayes Classifier}

\begin{table}[H]
    \begin{center}
    \begin{tabular}{|r||c|c|c|}
        \hline
        Description    & $\mu F$ & $\sigma F$ & $\delta\ \mu F$ \\
        \hline
        All            & 49.28  & 0.81 &  1.17 \\
        Just Stopwords & 47.29  & 0.76 & -0.82 \\
        Just Uncommon  & 48.34  & 1.11 &  0.23 \\
        Just Normalize & \textbf{49.44} & 1.00 & 1.33\\
        Baseline       & 48.11  & 0.74 & - \\
        \hline
    \end{tabular}
    \end{center}
    \caption{Performance of the Naive Bayes classifier using exhaustive
    cross-validation with 5 partitions}
    \label{table:bayes-results}
\end{table}

\begin{figure}[H]
    \begin{center}
    \includegraphics[scale=0.75]{figures/bayes_boxplot.png}
    \caption{This boxplot shows the F-score of each of ten folds in all five
    Naive Bayes experiments}
    \label{figure:bayes}
    \end{center}
\end{figure}

% TODO Student T-test

\subsection{Most Informative Features}
\label{ssection:informativefeatures}

The implementation create for this project has the ability to show which
features are the most discriminatory. The following is a list of the top 30
most informative features from one run of the Maximum Entropy classifier:

\begin{Verbatim}[fontsize=\footnotesize]
  -3.176 excited==True and label is 'neutral'
   2.328 lied==True and label is 'negative'
   2.279 availability==True and label is 'negative'
  -2.258 excited==True and label is 'negative'
  -2.180 EM_HAPPY==True and label is 'negative'
  -2.171 happy==True and label is 'neutral'
  -2.098 birthday==True and label is 'negative'
   2.016 shitty==True and label is 'negative'
  -1.836 8th==True and label is 'positive'
  -1.801 international==True and label is 'positive'
   1.789 embarrassing==True and label is 'negative'
  -1.762 went==True and label is 'negative'
   1.702 releases==True and label is 'negative'
  -1.672 love==True and label is 'negative'
   1.650 lack==True and label is 'negative'
   1.650 oneil==True and label is 'neutral'
  -1.638 fun==True and label is 'neutral'
   1.599 replay==True and label is 'negative'
  -1.585 great==True and label is 'neutral'
   1.577 asses==True and label is 'positive'
   1.570 garlic==True and label is 'negative'
  -1.564 pass==True and label is 'negative'
   1.560 den==True and label is 'negative'
   1.560 accepting==True and label is 'negative'
  -1.557 sure==True and label is 'negative'
   1.551 insensitive==True and label is 'negative'
   1.547 calvin==True and label is 'negative'
   1.532 jailed==True and label is 'negative'
   1.520 lookin==True and label is 'neutral'
   1.515 2c==True and label is 'negative'
\end{Verbatim}

Some of the features listed above make perfect sense, such as `excited' not
being associated with the neutral class and `shitty' being associated with the
negative class. However some of the most discriminating features appear in the
dataset just enough to get past the uncommon word cutoff, such as `calvin' and
`oneil'.

\subsection{Comparison to Other Systems}
\label{ssection:comparison}

Table \ref{table:comparison} shows the performance of the classifier produced
for this project compared to results of the workshop participants. The full
table (with out this project's results) can be found in table 9 of Nakov et al.
\cite{Nakov2013}.


\begin{table}[H]
    \begin{center}
    \begin{tabular}{|r|l|}
        \hline
        Team       & F \\
        \hline
        NRC-Canada & 69.02 \\
        GU-MLT-LT  & 65.27 \\
        teragram   & 64.86 \\
        BOUNCE     & 63.53 \\
        KLUE       & 63.06 \\
        AMI\&ERIC  & 62.55 \\
        FBM        & 61.17 \\
        AVAYA      & 60.84 \\
        SAIL       & 60.14 \\
        UT-DB      & 59.87 \\
        FBK-irst   & 59.76 \\
        nlp.cs.aueb.gr & 58.91 \\
        UNITOR     & 58.27 \\
        LVIC-LIMSI & 57.14 \\
        Unigon     & 56.96 \\
        NILC\_USP  & 56.31 \\
        DataMining & 55.52 \\
        ECNUCS     & 55.05 \\
        nlp.cs.aueb.gr & 54.73 \\
        ASVUniOfLeipzig & 54.56 \\
        SZTE-NLP   & 54.33 \\
        CodeX      & 53.89 \\
        Oasis      & 53.84 \\
        NTNU       & 53.23 \\
        UoM        & 51.81 \\
        \textbf{MaxEnt}& \textbf{50.96} \\
        SSA-UO     & 50.17 \\
        SenselyticTeam & 50.10 \\
        $\ldots$   & $\ldots$ \\
        9 more teams &  \textless\ 50 \\
        \hline
    \end{tabular}
    \end{center}
    \caption{Comparison of constrained classifiers sumitted to the SemEval-2013
        workshop. Sorted by F-score.}
    \label{table:comparison}
\end{table}
%TODO add naive bayes

\section{Conclusions and Future Work}
\label{section:conclusionsandfuturework}

\subsection{Conclusions}
\label{ssection:conclusions}

As shown in Table \ref{table:comparison} there is a lot of room for improvement
before this system is a completive solution. However, only simple methods have
been employed including: stopword removal of only 28 words, words with a
document frequency lower than one removed, and normalization of words, emoticons,
and OOV words. These simple preprocessing and feature selection methods in
combination with a Maximum Entropy classifier have demonstrated reasonable
results.

\subsection{Future Work}
\label{ssection:futurework}

\subsubsection{Hashtags}
\label{sssection:hashtags}

Hashtags are a way that Twitter users tag a Tweet as part of a particular
conversation. Hashtags refer to proper nouns such as events, places, or people.
Hashtags can also denote modifiers (such as `\#sarcasm') to the sentiment of the
Tweet, or even raw sentiment (e.g. `\#yay').

While it is likely infeasible to retain Hashtags about proper nouns,
classification performance could be improved by including modifiers and
Hashtags with raw sentiment. The following Tweet (not from the dataset) is a
great example of a Tweet that Hashtags could help improve classification
accuracy.

\begin{verbatim}
    Well I have an ear infection #good #sarcasm
\end{verbatim}

While a simple approach could be to strip the octothorpe from the front of the
Hashtag (e.g. `\#good' $\rightarrow$ `good'), the semantic meaning of the two
are different. Hashtags often have message level meaning while simple words
have local context withing the sentence. Since the dataset created for the
SemEval-2013 workshop mostly contained examples of Hashtags used as proper
nouns, more annotated Tweets would be needed for training and testing.

As well Hashtags sometimes use CamelCase to create multi-word tags (e.g.
`\#SorryNotSorry'). It could prove worthwhile to attempt the decompose
CamelCase Hashtags into its component words during feature selection.

The top solution from the National Research Council of Canada made use of a
Hashtag lexicon that they developed between April of 2012 and December of the
same year \cite{Mohammad2013}. The final \textit{NRC Hashtag Sentiment Lexicon}
has more than 350,00 entries and is available at
\href{http://www.saifmohammad.com/WebDocs/NRC-Hashtag-Sentiment-Lexicon-v0.1.zip}{Saif
Mohammad's website}.

\subsubsection{Users}
\label{sssection:users}

Nearly all Twitter users have published more than one Tweet. A more advanced
classifier could establish a particular users propensity to publishing a
prevailing sentiment in all of their Tweets.

As well, some Twitter users may be frequently be referred to with a particular
sentiment. For a hypothetical example, Tweets addressed to Emma Watson
(`@EmWatson') are more likely to have a positive sentiment, while Tweets to
Barack Obama (`@BarackObama') may contain extremely mixed sentiment.

A challenge that this method could introduce is the introduction of a temporal
element since people's outlooks, and therefore sentiment's in Tweets, will
change over time. As well as public perception of a individual changes, the
prevailing sentiment my change as well.

\subsubsection{Negation}
\label{sssection:negation}

One problem with the current implementation is that it lack support for
negation. In it's current form the classifier would likely misunderstand the
sequence ``not perfect'' since `not' is a vaguely negative word and `perfect'
is a very positive word, it would likely be classified as positive when in fact
the sentiment is quite clearly negative. A common approach to this problem is
to add a suffix (such as "\_NEG") to word that appear between a negation word
and a section delimiting punctuation mark (such as `,', `.', `:', `;', `!',
`?'). By separating negated features from their non-negated instances, features
can do a better job of distinguishing classes.

\subsubsection{N-gram Features}
\label{sssection:ngram}

One way that many of the other bag-of-words solutions capture word order is by
creating features consisting of sequences of words called bigrams. Adding
bigram features adds a large number of features which can add complexity to
training. This concept can also be extended to sequences of length n, called
ngrams.

The added complexity associated with ngrams did not deter the National Research
Council of Canada from using ngrams of length 1, 2, 3, and 4 as well as
`non-contiguous' ngrams which are described as `ngrams with one token replaced
by *' \cite{Mohammad2013}.

\subsubsection{Employing Sentiment Lexicon}
\label{sssection:sentimentlexicon}

Most of the more complex solutions submitted to the SemEval-2013 workshop made
uses of sentiment lexicons that capture a term's polarity. These lexicons, such
as \href{http://sentiwordnet.isti.cnr.it/}{SentiWordNet}, express the positive,
neutral, and negative sentiment of each homonym.

\subsubsection{Support Vector Machine}
\label{sssection:svm}

Many of the top solutions submitted to the workshop such as NRC-Canada
\cite{Mohammad2013} made use of a support vector machine for classification.
This indicates that when used effectively a SVM solution can out-perform a
Maximum Entropy solution. The KLUE team on the other hand found that their
Maximum Entropy solution outperform their linear SVM solution
\cite{Proisl2013}.

\section{References}
\label{section:referances}

\nocite{*}
\bibliographystyle{ieeetr}
\bibliography{paper}

\section{Appendices}
\label{section:appendices}

\subsection{Running the Implementation}
\label{ssection:runningtheimplementation}

To acquire your own copy of the implementation discussed above. The following
instructions are written for Ubuntu 14.04 GNU/Linux, however it should be able
to run on any Unix-like system.

\begin{enumerate}
    \item \textbf{Download Source} - The source code is hosted on github.com
        and can be viewed
        \href{https://github.com/hockeybuggy/semantic\_eval}{here}. To download a zip
        file and open it:

\begin{Verbatim}[fontsize=\footnotesize]
wget https://github.com/hockeybuggy/semantic_eval/archive/master.zip
unzip master.zip
cd master
\end{Verbatim}

% TODO test this and make a dist branch

    \item \textbf{Install Prerequisites} - If \textit{python 2.7},
        \textit{pip}, and \textit{flex} are not installed, they can be
        installed easily via a package manger:

\begin{Verbatim}[fontsize=\footnotesize]
sudo apt-get install python2.7 pip flex
\end{Verbatim}

Then various required python packages can be installed with:

\begin{Verbatim}[fontsize=\footnotesize]
sudo pip install numpy nltk
\end{Verbatim}

    \item \textbf{Compile Tokenizer} - The tokenizer is written in C for
        performance reasons and can be compiled with:
\begin{Verbatim}[fontsize=\footnotesize]
make
\end{Verbatim}

    \item \textbf{Run the Program} - The classifier can be executed in a number
        of ways. There are two python programs that are executable:
        `conductor.py' and `crossval.py'. These two programs run the classifier
        in a simple fashion or with cross validation. There are a large number
        of command line arguments that can be viewed with:

\begin{Verbatim}[fontsize=\footnotesize]
python conductor -h
\end{Verbatim}

Since the command line arguments can be cumbersome many of the experiments can
be run via the make file:

\begin{Verbatim}[fontsize=\footnotesize]
make tiny            # Quick running with only 100 examples
make crossfolds-m000 # The MaxEnt baseline with cross-validation
make crossfolds-m111 # MaxEnt with cross-validation & feature selection
make crossfolds-b000 # The Naive Bayes baseline with cross-validation
make crossfolds-b111 # Naive Bayes with cross-val & feature selection
make crossfolds      # Run all of the cross-validation experiments
\end{Verbatim}

\end{enumerate}

\subsection{Stopword list}
\label{ssection:stopwords}

The list of stopwords was acquired \href{www.ranks.nl/stopwords}{here} and is
as follows:

a, about, an, are, as, at, be, by, com, for, from, how, in, is, it, of, on, or,
that, the, this, to, was, what, when, where, who, will, with

\end{document}

